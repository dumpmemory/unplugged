\ifx\wholebook\relax \else

\documentclass[b5paper]{article}

\usepackage[nomarginpar
  %, margin=.5in
]{geometry}

\addtolength{\oddsidemargin}{-0.05in}
\addtolength{\evensidemargin}{-0.05in}
\addtolength{\textwidth}{0.1in}

\usepackage[en]{../../prelude}

\setcounter{page}{1}

\begin{document}

\title{Answers}

\author{LIU Xinyu
\thanks{{\bfseries LIU Xinyu} \newline
  Email: liuxinyu95@gmail.com \newline}
  }

\maketitle
\fi

\markboth{Answers}{Mathematics of programming}

\chapter*{Answers}
\phantomsection  % so hyperref creates bookmarks
\addcontentsline{toc}{chapter}{Answers}

\shipoutAnswer

\section{Infinity}

\begin{enumerate}

\item{In chapter 1, we realized Fibonacci numbers by folding. How to define Fibonacci numbers as potential infinity with $iterate$?}

\[
F = (fst \circ unzip)\ (iterate\ ((m, n) \mapsto (n, m + n))\ (1, 1))
\]

For example $take\ 100\ F$ gives the first 100 Fibonacci numbers

\item{Define $iterate$ by folding.}

Consider the infinite stream $iterate\ f\ x$. After applying $f$ to each element, and prepend $x$ as the first one, we obtain this infinite stream again. Based on this fact, we can define it as:

\[
iterate\ f\ x = x : foldr (y\ ys \mapsto (f\ y):ys)\ []\ (iterate\ f\ x)
\]

For example:

\begin{lstlisting}
take 10 $ iter (+1) 0
[0,1,2,3,4,5,6,7,8,9]
\end{lstlisting} %$

\item{Use the definition of the fixed point in chapter 4, prove $Stream$ is the fixed point of $StreamF$.}

Let $A' = \mathbf{StreamF}\ E\ A$, then apply it to itself repeatedly. We call this result $\mathbf{Fix}\ (\mathbf{StreamF}\ E)$

\bre
\mathbf{Fix}\ (\mathbf{StreamF}\ E) & = &
    \mathbf{StreamF}\ E\ (\mathbf{Fix}\ (\mathbf{StreamF}\ E)) & \text{definition of fixed point} \\
 & = & \mathbf{StreamF}\ E\ (\mathbf{StreamF}\ E\ (...)) & \text{expand recursively} \\
 & = & \mathbf{Stream}\ E\ (\mathbf{Stream}\ E\ (...)) & \text{change name} \\
 & = & \mathbf{Stream}\ E & \text{reverse of $Stream$} \\
\ere

Therefore, $Stream$ is the fixed point of $StreamF$.

\item{Define $unfold$.}

We often use $Maybe$ to define the terminate condition:

\begin{lstlisting}
unfold :: (b -> Maybe (a, b)) -> (b -> [a])
unfold f b = case f b of
                Just (a, b') -> a : unfold f b'
                Nothing -> []
\end{lstlisting}

\item{The fundamental theorem of arithmetic states that, any integer greater than 1 can be unique represented as the product of prime numbers. Given a text $T$, and a string $W$, does any permutation of $W$ exist in $T$? Solve this programming puzzle with the fundamental theorem and the stream of prime numbers.}

Our idea is to map every unique character to a prime number, for example, a $\to$ 2, b $\to$ 3, c $\to$ 5, ... Given any string $W$, no matter it contains repeated characters or not, we can convert it to a product of prime numbers:

\[
F = \prod p_c , c \in W
\]

We call it the number theory finger print $F$ of string $W$. When $W$ is empty, we define its finger print as 1. Because multiplication of integers is commutative, the finger print is same for all permutations of $W$, and according to the fundamental theorem of arithmetic, the finger print is unique. We can develop an elegant solution based on this: First, we calculate $F$ of string $W$, then slide a window of length $|W|$ along $T$ from left to right. When start, we also need to compute the finger print within this window of $T$, and compare it with $F$. If they are equal, it means $T$ contains some permutation of $W$. Otherwise, we slide the window to the right by a character. We can easily compute the updated finger print value for this new window position: divide the product by the prime number of the character slides out, and multiply the prime number of the character slides in. Whenever the finger print equals to $F$, we find a permutation. In order to map different characters to prime numbers, we can use sieve of Eratosthenes to generate a series of prime numbers. Below is the example algorithm accordingly.

\begin{algorithmic}
\Function{contains?}{$W, T$}
  \State $P \gets ana \ era \ [2, 3, ...]$ \Comment{prime numbers}
  \If{$W = \phi$}
    \State \Return True
  \EndIf
  \If{$|T| < |W|$}
    \State \Return False
  \EndIf
  \State $\displaystyle m \gets \prod P_c, c \in W$
  \State $\displaystyle m' \gets \prod P_c, c \in T[1...|W|]$
  \For{$i \gets |W| + 1$ to $|T|$}
    \If{$m = m'$}
      \State \Return True
    \EndIf
    \State $m' \gets m' \times P_{T_i} / P_{T_{i - |W|}} $
  \EndFor
  \State \Return $m = m'$
\EndFunction
\end{algorithmic}

\item{We establish the 1-to-1 correspondence between the rooms and guests. For guest $i$ in group $j$, which room number should be assigned? Which guest in which group will live in room $k$?}

Use the convention to count from zero, and use pair $(i, j)$ to denote the $j$-th guest in the $i$-th group. Let us list the first several guests and their rooms:

%\begin{adjustbox}{max width=\textwidth}
\btab{c|c|c|c|c|c|c|c|c|c|c|c}
$(i, j)$ & (0, 0) & (0, 1) & (1, 0) & (2, 0) & (1, 1) & (0, 2) & (0, 3) & (1, 2) & (2, 1) & (3, 0) & ... \\
\hline
$k$ & 0 & 1 & 2 & 3 & 4 & 5 & 6 & 7 & 8 & 9 & ... \\
\hline
$i + j$ & 0 & 1 & 1 & 2 & 2 & 2 & 3 & 3 & 3 & 3 & ... \\
\etab
%\end{adjustbox}

Writing down the values $i+j$, we can find the pattern. There are 1 instance of number 0, 2 instances of number 1, 3 instances of number 2, 4 instances of number 3, ... They are exactly the triangle numbers found by Pythagoreans. Let $m = i + j$, there are total $\dfrac{m(m + 1)}{2}$ grid points along the diagonals on the left-bottom side of a given grid.

For the diagonal where this point belongs to, if $m$ is odd, then the room number increases along the left up direction,  $i$ increases, and $j$ decreases; if $m$ is even, then the direction is right-bottom. Summarize the two cases gives the following result:

\[
k = \dfrac{m(m + 1)}{2} + \begin{cases} m - j: \text{$m$ is odd} \\
j: \text{$m$ is even} \\
\end{cases}
\]

Further, we can use $(-1)^m$ to simplify the conditions:

\[
k = \dfrac{m(m + 2) + (-1)^m (2j - m)}{2}
\]

\item{For Hilbert's Grand hotel, there are multiple solutions for the problem on the third day. Can you give a different numbering scheme based on the cover page of the book {\em Proof without word}?}

\begin{figure}[htbp]
\centering
\begin{tikzpicture}
  \draw[step=1, very thin, gray] (0, 0) grid (5, 5);
  \draw[->] (-0.25, 0) -- (6, 0) coordinate (x axis);
  \draw[->] (0, -0.25) -- (0, 6) coordinate (y axis);
  \foreach \x in {0, 1, 2, 3, 4, 5}
    \path (\x, -0.25) node[left] {\x};
  \foreach \y in {1, 2, 3, 4, 5}
    \path (-0.25, \y) node[below] {\y};
  \foreach \i / \x / \y in {0/0/0, 1/1/0, 2/1/1, 3/0/1, 4/0/2, 5/1/2, 6/2/2, 7/2/1, 8/2/0, 9/3/0, 10/3/1}{
    \path (\x, \y) coordinate (N\i);
    \fill (N\i) circle (1pt) node[above right=3pt of N\i] {\i};
  }
  \foreach \i in {0,...,9} {
    \pgfmathsetmacro{\j}{\i+1}
    \draw[-latex, thick] (N\i) to (N\j);
  }
\end{tikzpicture}
\caption{Another numbering scheme for infinity of infinity}
\label{fig:NNtoN2}
\end{figure}

As shown in figure \ref{fig:NNtoN2}, we count along the gnomon shaped path. There are odd number of grid points along every gnomon.

\item{Let $x = 0.9999....$, then $10x = 9.9999...$, subtract them gives $10x - x = 9$. Solving this equation gives $x = 1$. Hence $1 = 0.9999...$. Is this proof correct?}

Yes, it's correct.

\item{Light a candle between two opposite mirrors, what image can you see? Is it potential or actual infinity?}

The candle image reflects between the two mirrors endlessly, generates infinite many images. If we consider the speed of light is limited, then it is potential infinity from physics viewpoint.

\end{enumerate}

\section{Paradox}

\begin{enumerate}
\item{We can define numbers in natural language. For example ``the maximum of two digits number'' defines 99. Define a set containing all numbers that cannot be described within 20 words. Consider such an element: ``The minimum number that cannot be described within 20 words''. Is it a member of this set?}

This is an instance of Russell's paradox. Whether it is a member, all lead to contradiction.

\item{``The only constant is change'' said by Heraclitus. Is this Russell's paradox?}

Yes, this is an instance of Russell's paradox.

\item{Is the quote saying by Socrates (the beginning of chapter 7) Russell's paradox?}

Yes, it is an instance of Russell's paradox.

\item{Translate Fermat's last theorem into a TNT string.}

We need define power operation first.

\[\begin{cases}
\forall a: e(a, 0) = S0 & \text{0-th power is 1} \\
\forall a: \forall b: e(a, Sb) = a \cdot e(a, b) & \text{recursion} \\
\end{cases}\]

We can then define Fermat's last theorem atop it.

\[
\forall d: \lnot \exists a: \exists b: \exists c: \lnot (d = 0 \lor d = S0 \lor d = SS0) \to e(a, d) + e(b, d) = e(c, d)
\]

\item{Prove the associative law of addition with TNT reasoning rules.}

Surprisingly, we can prove every theorem below:

\bre
a + b + 0 & = & a + (b + 0) \\
a + b + S0 & = & a + (b + S0) \\
a + b + SS0 & = & a + (b + SS0) \\
... \\
\ere

For example:

\bre
a + b + 0 = a + b = a + (b + 0)
\ere

And:

\bre
a + b + SS0 & = & SS(a + b + 0) \\
 & = & SS(a + b) \\
 & = & a + SSb \\
 & = & a + (b + SS0) \\
\ere

However, we cannot prove: $\forall c: a + b + c = a + (b + c)$.

To do that, we has to introduce mathematical induction.

\item{Prove that $\forall a: (0 + a) = a$ with the newly added rule of induction.}

First for the case of 0:

\[
0 + 0 = 0
\]

Next suppose $(0 + a) = a$ holds, we have:

\bre
(0 + Sa) & = & S(0 + a) & \text{axiom 3} \\
  & = & Sa & \text{induction hypothesis} \\
\ere

From the rule of induction, we obtain: $\forall a: (0 + a) = a$

\end{enumerate}

\ifx\wholebook\relax \else
\begin{thebibliography}{99}

\bibitem{Lockhart2012}
Paul Lockhart. ``Measurement''. Belknap Press: An Imprint of Harvard University Press; Reprint edition 2014, ISBN: 978-0674284388

\end{thebibliography}

\expandafter\enddocument
%\end{document}

\fi
