\ifx\wholebook\relax \else

\documentclass[b5paper]{article}

\usepackage[nomarginpar
  %, margin=.5in
]{geometry}

\addtolength{\oddsidemargin}{-0.05in}
\addtolength{\evensidemargin}{-0.05in}
\addtolength{\textwidth}{0.1in}

\usepackage[en]{../../prelude}

\setcounter{page}{1}

\begin{document}

\title{Answers}

\author{LIU Xinyu
\thanks{{\bfseries LIU Xinyu} \newline
  Email: liuxinyu95@gmail.com \newline}
  }

\maketitle
\fi

\markboth{Answers}{Mathematics of programming}

\chapter*{Answers}
\phantomsection  % so hyperref creates bookmarks
\addcontentsline{toc}{chapter}{Answers}

\shipoutAnswer

\section{Paradox}

\begin{enumerate}
\item{We can define numbers in natural language. For example ``the maximum of two digits number'' defines 99. Define a set containing all numbers that cannot be described within 20 words. Consider such an element: ``The minimum number that cannot be described within 20 words''. Is it a member of this set?}

This is an instance of Russell's paradox. Whether it is a member, all lead to contradiction.

\item{``The only constant is change'' said by Heraclitus. Is this Russell's paradox?}

Yes, this is an instance of Russell's paradox.

\item{Is the quote saying by Socrates (the beginning of chapter 7) Russell's paradox?}

Yes, it is an instance of Russell's paradox.

\item{Translate Fermat's last theorem into a TNT string.}

We need define power operation first.

\[\begin{cases}
\forall a: e(a, 0) = S0 & \text{0-th power is 1} \\
\forall a: \forall b: e(a, Sb) = a \cdot e(a, b) & \text{recursion} \\
\end{cases}\]

We can then define Fermat's last theorem atop it.

\[
\forall d: \lnot \exists a: \exists b: \exists c: \lnot (d = 0 \lor d = S0 \lor d = SS0) \to e(a, d) + e(b, d) = e(c, d)
\]

\item{Prove the associative law of addition with TNT reasoning rules.}

Surprisingly, we can prove every theorem below:

\bre
a + b + 0 & = & a + (b + 0) \\
a + b + S0 & = & a + (b + S0) \\
a + b + SS0 & = & a + (b + SS0) \\
... \\
\ere

For example:

\bre
a + b + 0 = a + b = a + (b + 0)
\ere

And:

\bre
a + b + SS0 & = & SS(a + b + 0) \\
 & = & SS(a + b) \\
 & = & a + SSb \\
 & = & a + (b + SS0) \\
\ere

However, we cannot prove: $\forall c: a + b + c = a + (b + c)$.

To do that, we has to introduce mathematical induction.

\item{Prove that $\forall a: (0 + a) = a$ with the newly added rule of induction.}

First for the case of 0:

\[
0 + 0 = 0
\]

Next suppose $(0 + a) = a$ holds, we have:

\bre
(0 + Sa) & = & S(0 + a) & \text{axiom 3} \\
  & = & Sa & \text{induction hypothesis} \\
\ere

From the rule of induction, we obtain: $\forall a: (0 + a) = a$

\end{enumerate}

\ifx\wholebook\relax \else
\begin{thebibliography}{99}

\bibitem{Lockhart2012}
Paul Lockhart. ``Measurement''. Belknap Press: An Imprint of Harvard University Press; Reprint edition 2014, ISBN: 978-0674284388

\end{thebibliography}

\expandafter\enddocument
%\end{document}

\fi
