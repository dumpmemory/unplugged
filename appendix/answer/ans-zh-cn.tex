\ifx\wholebook\relax \else

\documentclass[UTF8]{article}

\usepackage[nomarginpar
  %, margin=.5in
]{geometry}

\addtolength{\oddsidemargin}{-0.05in}
\addtolength{\evensidemargin}{-0.05in}
\addtolength{\textwidth}{0.1in}

\usepackage[cn]{../../prelude}

\setcounter{page}{1}

\begin{document}

\title{参考答案}

\author{刘新宇
\thanks{{\bfseries 刘新宇} \newline
  Email: liuxinyu95@gmail.com \newline}
  }

\maketitle
\fi

\markboth{参考答案}{编程中的数学}

\chapter*{参考答案}
\phantomsection  % so hyperref creates bookmarks
\addcontentsline{toc}{chapter}{参考答案}

\shipoutAnswer

\section{融合}

\section{无穷}

\begin{enumerate}

\item{我们建立了房间和任意旅游团的客人间的一一映射。第$i$号旅游团的第$j$号客人应该入住几号房间?第$k$个房间里住了哪号旅游团的哪位客人?}

\item{希尔伯特旅馆第三天的故事的解法并不唯一,根据《无需语言的证明》一书的封面。试根据此图给出另一种编号方案?}

\item 令$x = 0.9999....$, 则$10x = 9.9999...$,做减法得$10x - x = 9$,解方程得$x = 1$。因此得到结论$1 = 0.9999...$。这一证明正确么?

正确

\item 在两个镜子中间点燃一支蜡烛,你看到了什么?这是潜无穷还是实无穷?

这支蜡烛在两个镜面间不断反射,产生无穷多的像。也许我们需要考虑光速是有限的,这样它在物理上仍然是潜无穷。

\end{enumerate}

\section{悖论}

\begin{enumerate}
\item {我们可以用语言定义数,例如“最大的两位数”定义了99。定义一个集合,是所有不能用20个以内的字描述的数字。考虑这样一个元素:“不能用20个以内的字描述的最小数”,它是否属于这个集合?}

这是一个罗素悖论,属于或不属于都将导致矛盾。

\item {“这个世界上唯一不变的是变化”——这句话是否是罗素悖论?}

是罗素悖论。

\item {本章开头苏格拉底的话是否是罗素悖论?}

是罗素悖论。

\item{尝试给出费马大定理的印符串。}

我们先要定义出幂运算。

\[\begin{cases}
\forall a: e(a, 0) = S0 & \text{任何数的0次幂为1} \\
\forall a: \forall b: e(a, Sb) = a \cdot e(a, b) & \text{递归} \\
\end{cases}\]

接着就可以定义费马大定理了:

\[
\forall d: \lnot \exists a: \exists b: \exists c: \lnot (d = 0 \lor d = S0 \lor d = SS0) \to e(a, d) + e(b, d) = e(c, d)
\]

\item{尝试用印符推理规则证明加法结合律。}

令人吃惊的是,我们可以证明下面的每一条定理:

\bre
a + b + 0 & = & a + (b + 0) \\
a + b + S0 & = & a + (b + S0) \\
a + b + SS0 & = & a + (b + SS0) \\
... \\
\ere

例如:

\bre
a + b + 0 = a + b = a + (b + 0)
\ere

以及:

\bre
a + b + SS0 & = & SS(a + b + 0) \\
 & = & SS(a + b) \\
 & = & a + SSb \\
 & = & a + (b + SS0) \\
\ere

但是却没有办法证明: $\forall c: a + b + c = a + (b + c)$。

为此必须引入数学归纳法。

\item{利用新加入的归纳规则证明$\forall a: (0 + a) = a$}

首先是0的情况:
\[
0 + 0 = 0
\]

然后假设$(0 + a) = a$成立,我们有:

\bre
(0 + Sa) & = & S(0 + a) & \text{公理3} \\
  & = & Sa & \text{归纳假设} \\
\ere

然后利用归纳规则,有:$\forall a: (0 + a) = a$

\end{enumerate}

\ifx\wholebook\relax \else
\begin{thebibliography}{99}

\bibitem{Lockhart2012}
[美] 保罗$\cdot$洛克哈特 著, 王凌云 译. ``度量——一首献给数学的情歌''. 人民邮电出版社. 2015, ISBN: 9787115393180

\end{thebibliography}

\expandafter\enddocument
%\end{document}

\fi
