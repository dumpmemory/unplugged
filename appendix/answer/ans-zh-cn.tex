\ifx\wholebook\relax \else

\documentclass[UTF8]{article}

\usepackage[nomarginpar
  %, margin=.5in
]{geometry}

\addtolength{\oddsidemargin}{-0.05in}
\addtolength{\evensidemargin}{-0.05in}
\addtolength{\textwidth}{0.1in}

\usepackage[cn]{../../prelude}

\setcounter{page}{1}

\begin{document}

\title{参考答案}

\author{刘新宇
\thanks{{\bfseries 刘新宇} \newline
  Email: liuxinyu95@gmail.com \newline}
  }

\maketitle
\fi

\markboth{参考答案}{编程中的数学}

\chapter*{参考答案}
\phantomsection  % so hyperref creates bookmarks
\addcontentsline{toc}{chapter}{参考答案}

\shipoutAnswer

\section{融合}

\section{无穷}

\begin{enumerate}

\item{第一章中,我们用叠加操作实现了斐波那契数列,如何用$iterate$定义斐波那契数列潜无穷?}

\[
F = (fst \circ unzip)\ (iterate\ ((m, n) \mapsto (n, m + n))\ (1, 1))
\]

例如$take\ 100\ F$

\item{用叠加操作定义$iterate$。}

我们考虑潜无穷流$iterate\ f\ x$,如果将$f$再次应用到每个元素上,并且在最前面添加一个$x$,得到的仍然是这个无穷流。基于这点我们可以定义:

\[
iterate\ f\ x = x : foldr (y\ ys \mapsto (f\ y):ys)\ []\ (iterate\ f\ x)
\]

例如:

\begin{lstlisting}
take 10 $ iter (+1) 0
[0,1,2,3,4,5,6,7,8,9]
\end{lstlisting} %$

\item{利用第四章中介绍的不动点定义,证明$Stream$是$StreamF$的不动点。}

令$A' = \mathbf{StreamF}\ E\ A$,然后不断递归地应用到自己,把此结果称为$\mathbf{Fix}\ (\mathbf{StreamF}\ E)$

\bre
\mathbf{Fix}\ (\mathbf{StreamF}\ E) & = &
    \mathbf{StreamF}\ E\ (\mathbf{Fix}\ (\mathbf{StreamF}\ E)) & \text{不动点的定义} \\
 & = & \mathbf{StreamF}\ E\ (\mathbf{StreamF}\ E\ (...)) & \text{递归展开} \\
 & = & \mathbf{Stream}\ E\ (\mathbf{Stream}\ E\ (...)) & \text{替换名称} \\
 & = & \mathbf{Stream}\ E & \text{反向用$Stream$的定义} \\
\ere

故$Stream$是$StreamF$的不动点。

\item{试定义反折叠$unfold$}

在实现时,通常使用$Maybe$来定义出一个结束条件:
\begin{lstlisting}
unfold :: (b -> Maybe (a, b)) -> (b -> [a])
unfold f b = case f b of
                Just (a, b') -> a : unfold f b'
                Nothing -> []
\end{lstlisting}

\item 数论中的算术基本定理说:任何一个大于1的整数都可以唯一地表示成若干素数的乘积。有一道编程趣题,要求判断一段文字$T$中,是否包含一个字符串$W$的某种排列。试利用算术基本定理,和素数流解决这道题目。

我们的思路是,将每一个不同字符对应到一个素数上去,a对应2,b对应3,c对应5……。这样任意给定一个字符串$W$,不管它是否包含重复的字符,我们都可以把它表示为素数的乘积:

\[
F = \prod p_c , c \in W
\]

我们称其为字符串$W$的数论指纹$F$。如果$W$是空串,我们规定它的指纹等于1。根据整数乘法的交换律,我们知道无论$W$怎样排列,其数论指纹都不变,并且根据算术基本定理,这个数论指纹是唯一的。现在我们就得到了一个特别简洁的解法:我们首先计算出$W$的数论指纹$F$,然后用一个长度为$|W|$的窗口沿着$T$从左向右滑动。一开始我们需要计算$T$在这个窗口内的数论指纹,并和$F$比较,如果相等就说明$T$包含$W$的某种排列。如果不等我们将这个窗口向右滑动一个字符。此时我们可以非常容易地计算新窗口内的数论指纹:只要把滑出的字符对应的素数除掉,再把滑入的字符对应的素数乘上就可以了。任何时候如果新窗口内的数论指纹等于$F$,就说明找到了一个排列。当然为了获得每个不同字符对应的素数,我们还要利用埃拉托斯特尼筛法产生一串素数。下面是一段示例算法:

\begin{algorithmic}
\Function{contains?}{$W, T$}
  \State $P \gets ana \ era \ [2, 3, ...]$ \Comment{素数序列}
  \If{$W = \phi$}
    \State \Return True
  \EndIf
  \If{$|T| < |W|$}
    \State \Return False
  \EndIf
  \State $\displaystyle m \gets \prod P_c, c \in W$
  \State $\displaystyle m' \gets \prod P_c, c \in T[1...|W|]$
  \For{$i \gets |W| + 1$ to $|T|$}
    \If{$m = m'$}
      \State \Return True
    \EndIf
    \State $m' \gets m' \times P_{T_i} / P_{T_{i - |W|}} $
  \EndFor
  \State \Return $m = m'$
\EndFunction
\end{algorithmic}

\item{我们建立了房间和任意旅游团的客人间的一一映射。第$i$号旅游团的第$j$号客人应该入住几号房间?第$k$个房间里住了哪号旅游团的哪位客人?}

按照本章约定,从0开始计数。用数偶$(i, j)$表示第$i$号旅游图的第$j$号客人。我们列出前面的几个客人和房间的对应关系

\btab{c|c|c|c|c|c|c|c|c|c|c|c}
$(i, j)$ & (0, 0) & (0, 1) & (1, 0) & (2, 0) & (1, 1) & (0, 2) & (0, 3) & (1, 2) & (2, 1) & (3, 0) & ... \\
\hline
$k$ & 0 & 1 & 2 & 3 & 4 & 5 & 6 & 7 & 8 & 9 & ... \\
\hline
$i + j$ & 0 & 1 & 1 & 2 & 2 & 2 & 3 & 3 & 3 & 3 & ... \\
\etab

如果同时写下$i+j$的值,我们发现规律是很明显的。共有1个0,2个1,3个2,4个3……这些恰恰是毕达哥拉斯发现的三角形数。记$m = i + j$,对于图中任意格点,它表明在这个点的左下方所有斜线上格点的数目为:$\dfrac{m(m + 1)}{2}$。

在这个点所在的斜线上,如果$m$是奇数则向左上前进,$i$增加、$j$减小;如果是偶数则向右下前进。综合起来,我们得到结果:

\[
k = \dfrac{m(m + 1)}{2} + \begin{cases} m - j: \text{$m$是奇数} \\
j: \text{$m$是偶数} \\
\end{cases}
\]

进一步,我们可以通过$(-1)^m$来简化这个结果:

\[
k = \dfrac{m(m + 2) + (-1)^m (2j - m)}{2}
\]

\item{希尔伯特旅馆第三天的故事的解法并不唯一,根据《无需语言的证明》一书的封面。试根据此图给出另一种编号方案?}

\begin{figure}[htbp]
\centering
\begin{tikzpicture}
  \draw[step=1, very thin, gray] (0, 0) grid (5, 5);
  \draw[->] (-0.25, 0) -- (6, 0) coordinate (x axis);
  \draw[->] (0, -0.25) -- (0, 6) coordinate (y axis);
  \foreach \x in {0, 1, 2, 3, 4, 5}
    \path (\x, -0.25) node[left] {\x};
  \foreach \y in {1, 2, 3, 4, 5}
    \path (-0.25, \y) node[below] {\y};
  \foreach \i / \x / \y in {0/0/0, 1/1/0, 2/1/1, 3/0/1, 4/0/2, 5/1/2, 6/2/2, 7/2/1, 8/2/0, 9/3/0, 10/3/1}{
    \path (\x, \y) coordinate (N\i);
    \fill (N\i) circle (1pt) node[above right=3pt of N\i] {\i};
  }
  \foreach \i in {0,...,9} {
    \pgfmathsetmacro{\j}{\i+1}
    \draw[-latex, thick] (N\i) to (N\j);
  }
\end{tikzpicture}
\caption{对无穷个无穷的另一种编号方案}
\label{fig:NNtoN2}
\end{figure}

如图\ref{fig:NNtoN2}所示,每次沿着折尺形前进,每个折尺上有奇数个点。

\item 令$x = 0.9999....$, 则$10x = 9.9999...$,做减法得$10x - x = 9$,解方程得$x = 1$。因此得到结论$1 = 0.9999...$。这一证明正确么?

正确

\item 在两个镜子中间点燃一支蜡烛,你看到了什么?这是潜无穷还是实无穷?

这支蜡烛在两个镜面间不断反射,产生无穷多的像。也许我们需要考虑光速是有限的,这样它在物理上仍然是潜无穷。

\end{enumerate}

\section{悖论}

\begin{enumerate}
\item {我们可以用语言定义数,例如“最大的两位数”定义了99。定义一个集合,是所有不能用20个以内的字描述的数字。考虑这样一个元素:“不能用20个以内的字描述的最小数”,它是否属于这个集合?}

这是一个罗素悖论,属于或不属于都将导致矛盾。

\item {“这个世界上唯一不变的是变化”——这句话是否是罗素悖论?}

是罗素悖论。

\item {本章开头苏格拉底的话是否是罗素悖论?}

是罗素悖论。

\item{尝试给出费马大定理的印符串。}

我们先要定义出幂运算。

\[\begin{cases}
\forall a: e(a, 0) = S0 & \text{任何数的0次幂为1} \\
\forall a: \forall b: e(a, Sb) = a \cdot e(a, b) & \text{递归} \\
\end{cases}\]

接着就可以定义费马大定理了:

\[
\forall d: \lnot \exists a: \exists b: \exists c: \lnot (d = 0 \lor d = S0 \lor d = SS0) \to e(a, d) + e(b, d) = e(c, d)
\]

\item{尝试用印符推理规则证明加法结合律。}

令人吃惊的是,我们可以证明下面的每一条定理:

\bre
a + b + 0 & = & a + (b + 0) \\
a + b + S0 & = & a + (b + S0) \\
a + b + SS0 & = & a + (b + SS0) \\
... \\
\ere

例如:

\bre
a + b + 0 = a + b = a + (b + 0)
\ere

以及:

\bre
a + b + SS0 & = & SS(a + b + 0) \\
 & = & SS(a + b) \\
 & = & a + SSb \\
 & = & a + (b + SS0) \\
\ere

但是却没有办法证明: $\forall c: a + b + c = a + (b + c)$。

为此必须引入数学归纳法。

\item{利用新加入的归纳规则证明$\forall a: (0 + a) = a$}

首先是0的情况:
\[
0 + 0 = 0
\]

然后假设$(0 + a) = a$成立,我们有:

\bre
(0 + Sa) & = & S(0 + a) & \text{公理3} \\
  & = & Sa & \text{归纳假设} \\
\ere

然后利用归纳规则,有:$\forall a: (0 + a) = a$

\end{enumerate}

\ifx\wholebook\relax \else
\begin{thebibliography}{99}

\bibitem{Lockhart2012}
[美] 保罗$\cdot$洛克哈特 著, 王凌云 译. ``度量——一首献给数学的情歌''. 人民邮电出版社. 2015, ISBN: 9787115393180

\end{thebibliography}

\expandafter\enddocument
%\end{document}

\fi
