\ifx\wholebook\relax \else

\documentclass[UTF8]{article}

\usepackage[cn]{../../prelude}

\setcounter{page}{1}

\begin{document}

\fi

\markboth{参考文献}{编程中的数学}
\phantomsection  % so hyperref creates bookmarks
\addcontentsline{toc}{chapter}{参考文献}

\begin{thebibliography}{99}

% 自然数
% =====================================

\bibitem{wiki-number}
维基百科. 古代计数系统的历史 [Z/OL]. (2022-07-12)[2022-07-12]. \url{https://en.wikipedia.org/wiki/History_of_ancient_numeral_systems}.

\bibitem{trip-to-number-kingdom}
卡尔文$\cdot$C$\cdot$克劳森. 数学旅行家:漫游数王国 [M]. 袁向东,袁钧,译. 上海:上海教育出版社,2001. %ISBN: 7-5320-7883-3/G $\cdot$ 7972

\bibitem{wiki-babylonian-num}
维基百科. 古巴比伦数字 [Z/OL]. (2021-11-27)[2022-7-12]. \url{https://en.wikipedia.org/wiki/Babylonian_numerals}.

\bibitem{M-Kline-2007}
M$\cdot$克莱因. 数学:确定性的丧失 [M]. 李宏魁,译. 湖南:湖南科学技术出版社,2007. %ISBN: 978-7-5357-1857-0
% Morris Kline ``Mathematics: The Loss of Certainty''. Oxford University Press, 1980.

\bibitem{GEB}
候世达. 哥德尔、埃舍尔、巴赫——集异壁之大成 [M]. 严勇,刘皓明,莫大伟,等译. 北京:商务印书馆, 1996. %ISBN: 978-7-100-01323-9

\bibitem{Bird97}
Richard Bird, Oege de Moor. Algebra of Programming [M]. University of Oxford, Prentice Hall Europe. 1997. %ISBN: 0-13-507245-X.

\bibitem{Gusen2014}
顾森. 浴缸里的惊叹 [M]. 北京:人民邮电出版社, 2014. %ISBN: 9787115355744

% 递归
% ==============================

\bibitem{HanXueTao16}
韩雪涛. 数学悖论与三次数学危机 [M]. 北京:人民邮电出版社, 2016. %ISBN: 9787115430434

\bibitem{StepanovRose15}
亚历山大\ A$\cdot$斯捷潘诺夫,丹尼尔\ E$\cdot$罗斯. 数学与泛型编程:高效编程的奥秘 [M]. 爱飞翔, 译. 北京:机械工业出版社, 2017. %ISBN: 9787111576587

\bibitem{MKlein1972}
莫里斯$\cdot$克莱因. 古今数学思想,第一册 [M]. 张理京, 等译. 上海:上海科学技术出版社, 2014. %ISBN: 9787547817179

\bibitem{Elements}
欧几里得. 几何原本 [M]. 兰纪正, 朱恩宽, 译,梁宗巨, 张毓新, 徐伯谦, 校订. 南京:译林出版社, 2014. %ISBN: 9787544750066

\bibitem{HanXueTao2009}
韩雪涛. 好的数学——“下金蛋”的数学问题 [M]. 长沙:湖南科学技术出版社, 2009. %ISBN: 9787535756725

\bibitem{Bezout-Identity}
维基百科. 贝祖等式 [Z/OL]. (2022-06-25)[2022-07-12]. \url{https://en.wikipedia.org/wiki/Bézout's_identity}.

\bibitem{LiuXinyu2017}
刘新宇. 算法新解 [M]. 北京:人民邮电出版社, 2017. %ISBN: 9787115440358

\bibitem{wiki-Turing}
维基百科. 艾伦$\cdot$图灵 [Z/OL]. (2022-07-11)[2022-07-12]. \url{https://en.wikipedia.org/wiki/Alan_Turing}.

\bibitem{Dowek2011}
吉尔$\cdot$多维克. 计算进化史:改变数学的命运 [M]. 劳佳, 译. 北京:人民邮电出版社, 2017. %ISBN: 9787115447579

\bibitem{SPJ1987}
Simon L. Peyton Jones. The implementation of functional programming language [M]. New York: Prentice Hall, 1987. %ISBN: 013453333X

% 抽象代数:群、环、域、伽罗瓦理论
% ==================================

\bibitem{HanXueTao2012}
韩雪涛. 好的数学——方程的故事 [M]. 长沙:湖南科学技术出版社, 2012. %ISBN: 9787535770066

\bibitem{Wiki-Galois-theory}
维基百科. 伽罗瓦理论 [Z/OL]. (2022-05-24)[2022-07-12]. \url{https://en.wikipedia.org/wiki/Galois_theory}.

\bibitem{Wiki-Galois}
维基百科. 埃瓦里斯特$\cdot$伽罗瓦 [Z/OL]. (2022-04-28)[2022-07-12]. \url{https://en.wikipedia.org/wiki/Évariste_Galois}.

\bibitem{Wiki-Rubik-Cube-group}
维基百科. 魔方群 [Z/OL]. (2022-01-20)[2022-07-12]. \url{https://en.wikipedia.org/wiki/Rubik's_Cube_group}.

\bibitem{ZhangHeRui1978}
张禾瑞. 近世代数基础 [M]. 北京:高等教育出版社, 1978. %ISBN: 9787040012224

\bibitem{Armstrong1988}
M.A. Armstrong. 群与对称(影印版) [M]. New York: Springer, 1988. %ISBN: 0387966757.

\bibitem{Wiki-Lagrange}
维基百科. 约瑟夫$\cdot$拉格朗日 [Z/OL]. (2022-06-10)[2022-07-12]. \url{https://en.wikipedia.org/wiki/Joseph-Louis_Lagrange}.

\bibitem{Wiki-FLT-proof}
维基百科. 费马小定理的证明 [Z/OL]. (2022-07-01)[2022-07-12]. \url{https://en.wikipedia.org/wiki/Proofs_of_Fermat's_little_theorem}.

\bibitem{Weil83}
安德烈$\cdot$韦伊. 数论——从汉穆拉比到勒让德的历史导引 [M]. Birkhäuser Boston, MA, 2006 %ISBN: 978-0-8176-4565-6

\bibitem{Wiki-Euler}
维基百科. 莱昂哈德$\cdot$欧拉 [Z/OL]. (2022-07-09)[2022-07-12]. \url{https://en.wikipedia.org/wiki/Leonhard_Euler}.

\bibitem{Wiki-Carmichael-number}
维基百科. 卡米歇尔数 [Z/OL]. (2022-6-23)[2022-07-12]. \url{https://en.wikipedia.org/wiki/Carmichael_number}.

\bibitem{Algorithms-DPV}
Sanjoy Dsgupta, Christos Papadimitriou, Umesh Vazirani. 算法概论(注释版) [M]. 钱枫, 邹恒明, 注释. 北京: 机械工业出版社, 2009. %ISBN: 9787111253617

\bibitem{Wiki-Miller-Rabin}
维基百科. 米勒——拉宾素数检验 [Z/OL]. (2022-6-20)[2022-07-12]. \url{https://en.wikipedia.org/wiki/Miller-Rabin_primality_test}.

\bibitem{Wiki-Noether}
维基百科. 埃米$\cdot$诺特 [Z/OL]. (2022-07-09)[2022-07-12]. \url{https://en.wikipedia.org/wiki/Emmy_Noether}.

\bibitem{ZhangPu2013}
章璞. 伽罗瓦理论:天才的激情 [M]. 北京: 高等教育出版社, 2013. %ISBN: 9787040372526

\bibitem{Stillwell1994}
John Stillwell. Galois Theory for Beginners  [J]. The American Mathematical Monthly, Vol. 101, No. 1 (Jan., 1994), pp. 22-2

\bibitem{Goodman2011}
Dan Goodman. An Introduction to Galois Theory [Z/OL]. (2022)[2022-07-12]. \url{https://nrich.maths.org/1422}.

\bibitem{MArtin}
Michael Artin. 代数(英文版,第二版) [M]. 北京: 机械工业出版社, 2011. %ISBN: 9787111367017

\bibitem{Weyl1952}
赫尔曼$\cdot$外尔. 对称 [M]. 冯承天, 陆继宗, 译. 北京: 北京大学出版社, 2018. %ISBN: 9787301291719

\bibitem{FengChengTian2019}
冯承天. 从一元一次方程到伽罗瓦理论(第二版) [M]. 上海: 华东师范大学出版社, 2019. %ISBN: 9787567587380

\bibitem{JieChengHao2021}
结城浩. 数学女孩. 5,伽罗瓦理论 [M]. 陈冠贵, 译. 北京: 人民邮电出版社, 2021. %ISBN: 9787115559623

% 范畴论
% =========================================

\bibitem{Dieudonne1987}
让$\cdot$迪厄多内. 当代数学,为了人类心智的荣耀 [M]. 沈用欢, 译. 上海: 上海教育出版社, 2000. %ISBN: 7532063062

\bibitem{Monad-Haskell-Wiki}
哈斯克尔维基. Monad [Z/OL]. (2021-08-01)[2022-07-12]. \url{https://wiki.haskell.org/Monad}.

\bibitem{Wiki-Eilenberg}
维基百科. 塞缪尔$\cdot$艾伦伯格 [Z/OL]. (2022-05-26)[2022-07-12]. \url{https://en.wikipedia.org/wiki/Samuel_Eilenberg}.

\bibitem{Wiki-Mac-Lane}
维基百科. 桑德斯$\cdot$麦克兰恩 [Z/OL]. (2022-06-19)[2022-07-12]. \url{https://en.wikipedia.org/wiki/Saunders_Mac_Lane}.

\bibitem{Simmons2011}
Harold Simmons. An introduction to Category Theory [M]. Cambridge University Press, 2011. %ISBN: 9780521283045

\bibitem{Wiki-Hoare}
维基百科. Tony Hoare [Z/OL]. (2022-07-07)[2022-07-12]. \url{https://en.wikipedia.org/wiki/Tony_Hoare}.

\bibitem{Wadler-1989}
Wadler Philip. Theorems for free! [C]. Functional Programming Languages and Computer Architecture, Association for Computing Machinery. 1989. pp. 347-359.

\bibitem{Milewski2018}
Bartosz Milewski. Category Theory for Programmers [Z/OL]. (2014-10-28)[2022-07-12]. \url{https://bartoszmilewski.com/2014/10/28/category-theory-for-programmers-the-preface/}.

\bibitem{PeterSmith2018}
Peter Smith. Category Theory - A Gentle Introduction [Z/OL]. (2018-01-01)[2022-07-12]. \url{http://www.academia.edu/21694792/A_Gentle_Introduction_to_Category_Theory_Jan_2018_version_}.

\bibitem{Wiki-Exponentials}
维基百科. Exponential Object [Z/OL]. (2021-12-12)[2022-07-12]. \url{https://en.wikipedia.org/wiki/Exponential_object}.

\bibitem{Manes-Arbib-1986}
Manes, E. G. and Arbib, M. A. Algebraic Approaches to Program Semantics [M]. Texts and Monographs in Computer Science. New York: Springer-Verlag, 1986.

\bibitem{Lambek-1968}
Lambek, J. A fixpoint theorem for complete categories [J]. Mathematische Zeischrift, 103, 1968. pp.151-161.

\bibitem{Haskell-foldable}
维基图书. Haskell/Foldable [Z/OL]. (2022-04-16)[2022-07-12]. \url{https://en.wikibooks.org/wiki/Haskell/Foldable}.

\bibitem{Mac-Lane-1998}
Mac Lane. Categories for working mathematicians [M]. New York: Springer-Verlag, 1998. %ISBN: 0387984038.

% 推理
% ===========================================

\bibitem{GLPJ-1993}
Andrew Gill, John Launchbury, Simon L. Peyton Jones. A Short Cut to Deforestation [J]. Functional programming languages and computer architecture, 1993. pp. 223-232.

\bibitem{Bird-2010}
Richard Bird. Pearls of Functional Algorithm Design [M]. Cambridge University Press, 2010. %ISBN: 978-0521513388.

\bibitem{Hinze-Harper-James-2010}
Ralf Hinze, Thomas Harper, Daniel W. H. James. Theory and Practice of Fusion [C]. 22nd international symposium of IFL (Implementation and application of functional languages), 2010. pp.19-37.

\bibitem{Takano-Meijer-1995}
Akihiko Takano, Erik Meijer. Shortcut Deforestation in Calculational Form [J]. Functional programming languages and computer architecture. 1995. pp. 306-313.

\bibitem{Knuth-TAOCP-2006}
Donald Knuth. The Art of Computer Programming, Volume 4, Fascicle 4: Generating All Trees [M]. MA: Addison-Wesley, 2006. %ISBN: 978-0321637130.

% 无穷
% ===========================================
\bibitem{De-linfini-2018}
让-皮埃尔$\cdot$卢米涅, 马克$\cdot$拉雪茨-雷. 从无穷开始——科学的困惑与疆界 [M]. 孙展, 译. 北京: 人民邮电出版社, 2018. %ISBN: 9787115479198

\bibitem{Noguchi2007}
野口哲也. 数学原来可以这样学 [M]. 刘慧, 韩丽红, 译. 长沙: 湖南人民出版社, 2014. %ISBN: 9787556100897
% Tetsunori Noguchi. SUGAKUTEKI SENSE GA MINITUKU RENSHUCHO.

\bibitem{Wikipedia-Googol}
维基百科. Googol [Z/OL]. (2022-06-21)[2022-07-12]. \url{https://en.wikipedia.org/wiki/Googol}.

\bibitem{Wikipedia-Zeno}
维基百科. Zeno's Paradoxes [Z/OL]. (2022-07-06)[2022-07-12]. \url{https://en.wikipedia.org/wiki/Zeno's_paradoxes}.

\bibitem{GCH}
张锦文, 王雪生. 连续统假设 [M]. 沈阳:辽宁教育出版社, 1988. %ISBN: 7-5382-0436-9/G$\cdot$445

\bibitem{Courant1969}
R$\cdot$柯朗, H$\cdot$罗宾.. 什么是数学:对思想和方法的基本研究(中文版第四版)[M]. I$\cdot$斯图尔特, 修订, 左平, 张饴慈, 译. 上海: 复旦大学出版社, 2017. %ISBN: 9787309086232

% Richard Courant, Herbert Robbins , Reviewed by Ian Stewart. ``What Is Mathematics? An Elementary Approach to Ideas and Methods 2nd Edition''. Oxford University Press,  1996, ISBN: 978-0195105193.

\bibitem{Poincare1}
彭加勒. 科学与假设 [M]. 李醒民, 译. 北京: 商务印书馆, 2006. %ISBN: 978-7-100-04796-8

% 悖论
% ===========================================
\bibitem{Gatys-2015}
Leon A. Gatys, Alexander S. Ecker, Matthias Bethge. A Neural Algorithm of Artistic Style [C]. IEEE Conference on Computer Vision and Pattern Recognition (CVPR), 2017.

\bibitem{GuSen-2012}
顾森. 思考的乐趣——Matrix67数学笔记 [M]. 北京: 人民邮电出版社, 2012. %ISBN: 9787115275868

\bibitem{SICP}
Harold Abelson, Gerald Jay Sussman, Julie Sussman. 计算机程序的构造和解释(原书第二版) [M]. 裘宗燕, 译. 北京: 机械工业出版社, 2004. %ISBN: 7-111-13510-5

\bibitem{Poincare2}
彭加勒. 科学的价值 [M]. 李醒民, 译. 北京: 商务印书馆, 2010. %ISBN: 978-7-100-07045-4

\bibitem{Ried-1996}
康斯坦丝·瑞德. 希尔伯特:数学界的亚历山大 [M]. 袁向东, 李文林, 译. 上海: 上海科学技术出版社, 2018. %ISBN: 978-7-5478-4088-7

% 附录答案
% ========================================

\bibitem{Lockhart2012}
保罗$\cdot$洛克哈特. 度量——一首献给数学的情歌 [M]. 王凌云, 译. 北京: 人民邮电出版社, 2015. %ISBN: 9787115393180

\end{thebibliography}

\ifx\wholebook\relax \else

\expandafter\enddocument
%\end{document}

\fi
