\ifx\wholebook\relax \else

\documentclass[UTF8]{article}

\usepackage[en]{../../prelude}

\setcounter{page}{1}

\begin{document}

\title{Category}

\author{LIU Xinyu
\thanks{{\bfseries LIU Xinyu} \newline
  Email: liuxinyu99@hotmail.com \newline}
  }

\maketitle
\fi

\markboth{Category}{Mathematics of Programming}

\chapter*{Uniqueness of product and coproduct}
\phantomsection  % so hyperref creates bookmarks
\addcontentsline{toc}{chapter}{Uniqueness of product and coproduct}

Are product and coproduct unique? The following theorem answers this question.

\begin{lemma}
\normalfont
For any pair of object $A$, $B$ of category $\pmb{C}$, let the objects and arrows in below diagram

\begin{center}
\begin{tikzpicture}
  \matrix (m) [matrix of math nodes,
               row sep=2em, column sep=2em, minimum width=2em]{
    P &   &  &   & I \\
      & A &  & A &   \\
      & B &  & B &   \\
    Q &   &  &   & J \\};
  \path[-stealth]
    % left
    (m-1-1) edge node [above] {$p_A$} (m-2-2)
    (m-1-1) edge node [below] {$p_B$} (m-3-2)
    (m-4-1) edge node [above] {$q_A$} (m-2-2)
    (m-4-1) edge node [below] {$q_B$} (m-3-2)
    % right
    (m-2-4) edge node [above] {$i_A$} (m-1-5)
    (m-2-4) edge node [above] {$j_A$} (m-4-5)
    (m-3-4) edge node [below] {$i_B$} (m-1-5)
    (m-3-4) edge node [below] {$j_B$} (m-4-5);
\end{tikzpicture}
\end{center}

be a pair of

\begin{center}
product \quad \quad \quad coproduct
\end{center}

wedges, then

\[
P, Q \quad \quad \quad I, J
\]

are isomorphic wedges. There are unique arrows:

\begin{center}
\begin{tikzpicture}
  \matrix (m) [matrix of math nodes,
               row sep=2em, column sep=3em, minimum width=2em]{
    P & & I \\
    Q & & J \\};
  \path[-stealth]
    % left
    (m-1-1.south west) edge node [left] {$f$} (m-2-1.north west)
    (m-2-1.north east) edge node [right] {$g$} (m-1-1.south east)
    % right
    (m-1-3.south west) edge node [left] {$f$} (m-2-3.north west)
    (m-2-3.north east) edge node [right] {$g$} (m-1-3.south east);
\end{tikzpicture}
\end{center}

Such that:

\[
\begin{array}{lcl}
  \begin{cases}
    p_A = q_A \circ f & p_B = q_B \circ f \\
    q_A = p_A \circ g & q_B = p_B \circ g
  \end{cases}
  & \quad &
  \begin{cases}
    i_A = g \circ j_A & i_B = g \circ j_B \\
    j_A = f \circ i_A & j_B = f \circ i_B
  \end{cases}
\end{array}
\]

Where $f$ and $g$ are inverse pair of isomorphisms.
\end{lemma}

\begin{proof}
We only prove the left side part. The right side can be proved in a similar way. Given $A$, $B$, Object $Q$ and the pair $q_A, q_B$ form a product wedge. Object $P$ and the pair $p_A, p_B$ form another wedge. From the definition of product, there is a unique mediator $f$ satisfying:

\[
p_A = q_A \circ f \quad \text{and} \quad p_B = q_B \circ f
\]

By reversing the role of $P$ and $Q$ (Let $P$ be product, and $Q$ be arbitrary wedge), we have:

\[
q_A = p_A \circ g \quad \text{and} \quad q_B = p_B \circ g
\]

Hence we have:

\[
\begin{cases}
p_A \circ g \circ f = q_A \circ f = p_A \\
p_B \circ g \circ f = q_B \circ f = p_B
\end{cases}
\]

and:

\[
\begin{cases}
q_A \circ f \circ g = p_A \circ g = q_A \\
q_B \circ f \circ g = p_B \circ g = q_B \\
\end{cases}
\]

Therefore:

\[
g \circ f = id_P \quad \quad \quad f \circ g = id_Q
\]

\end{proof}

This proved if two objects have product (or coproduct), then it is unique.

\chapter*{Cartesian product and disjoint union of sets form product and coproduct}
\phantomsection  % so hyperref creates bookmarks
\addcontentsline{toc}{chapter}{Cartesian product and disjoint union of sets form product and coproduct}

\begin{proof}
We prove this by construction. The Cartesian product $A \times B$ contains all combinations from the two sets.

\[
\{(a, b) | a \in A, b \in B\}
\]

We define two special arrows (functions) as $p_A$ and $p_B$:

\[
\begin{cases}
fst\ (a, b) = a \\
snd\ (a, b) = b
\end{cases}
\]

Consider an arbitrary wedge

\begin{center}
\begin{tikzpicture}
  \matrix (m) [matrix of math nodes,
               row sep=2em, column sep=2em, minimum width=2em]{
      & A  \\
    X &    \\
      & B  \\};
  \path[-stealth]
    (m-2-1) edge node[above] {$p$} (m-1-2)
    (m-2-1) edge node[below] {$q$} (m-3-2);
\end{tikzpicture}
\end{center}

Where $p\ x = a, q\ x = b, x \in X$. For example, let $X$ be $Int$, $A$ be $Int$, and $B$ be $Bool$, the two functions $p$ and $q$ are defined as:

\[
\begin{cases}
p(x) = -x \\
q(x) = even(x)
\end{cases}
\]

Such that $p$ negates an integer, and $q$ examines if it is even. We define the function $X \arrowto{m} A \times B$ as below:

\[
m(x) = (a, b)
\]

For this example, we have:

\[
m(x) = (-x, even(x))
\]

Such that, the following diagram commutes:

\begin{center}
\begin{tikzpicture}
  \matrix (m) [matrix of math nodes,
               row sep=2em, column sep=2em, minimum width=2em]{
      & A \\
    X & A \times B \\
      & B \\};
  \path[-stealth]
    % left
    (m-2-1) edge node [above] {$p$} (m-1-2)
    (m-2-1) edge node [below] {$q$} (m-3-2)
    (m-2-1) edge node [above] {$m$}   (m-2-2)
    (m-2-2) edge node [right] {$fst$} (m-1-2)
    (m-2-2) edge node [right] {$snd$} (m-3-2);
\end{tikzpicture}
\end{center}

Let us verify it:

\[
\begin{array}{rcll}
(fst \circ m)(x) & = & fst\ m(x) & \text{function composition} \\
                 & = & fst\ (a, b) & \text{definition of $m$} \\
                 & = & a & \text{definition of $fst$} \\
                 & = & p(x) & \text{reverse of $p$}
\end{array}
\]

and:

\[
\begin{array}{rcll}
(snd \circ m)(x) & = & snd\ m(x) & \text{function composition} \\
                 & = & snd\ (a, b) & \text{definition of $m$} \\
                 & = & b & \text{definition of $snd$} \\
                 & = & q(x) & \text{reverse of $q$}
\end{array}
\]

For the above example, we have:

\[
\begin{cases}
(fst \circ m)(x) = fst\ (-x, even(x)) = -x = p(x) \\
(snd \circ m)(x) = snd\ (-x, even(x)) = even(x) = q(x)
\end{cases}
\]

We also need prove the uniqueness of $m$. Suppose there exists another function $x \arrowto{h} A \times B$, such that the diagram also commutes:

\[
fst \circ h = p \quad \text{and} \quad snd \circ h = q
\]

We have:

\[
\begin{array}{rcll}
(a, b) & = & (p(x),\ q(x)) & \text{definition of $p, q$} \\
       & = & ((fst \circ h)(x),\ (snd \circ h)(x)) & \text{commute} \\
       & = & (fst\ h(x),\ snd\ h(x)) & \text{function composition} \\
       & = & (fst\ (a, b),\ snd\ (a, b)) & \text{reverse of $fst, snd$}
\end{array}
\]

Hence $h(x) = (a, b) = m(x)$, which proves the uniqueness of $m$.

Next we prove the coproduct part. The elements in the disjoint union $A + B$ have two types. One comes from $A$ as $(a, 0)$, the other comes from $B$ as $(b, 1)$. We can define two special arrows (functions) as $i_A$ and $i_B$:

\[
\begin{cases}
left(a) = (a, 0) \\
right(b) = (b, 1)
\end{cases}
\]

Consider the wedge of an arbitrary set $X$:

\begin{center}
\begin{tikzpicture}
  \matrix (m) [matrix of math nodes,
               row sep=2em, column sep=2em, minimum width=2em]{
    A & \\
      & X \\
    B & \\};
  \path[-stealth]
    (m-1-1) edge node [above] {$p$} (m-2-2)
    (m-3-1) edge node [below] {$q$} (m-2-2);
\end{tikzpicture}
\end{center}

We define arrow $A + B \arrowto{m} X$ as below:

\[
\begin{cases}
m\ (a, 0) = p(a) \\
m\ (b, 1) = q(b)
\end{cases}
\]

Such that the following diagram commutes:

\begin{center}
\begin{tikzpicture}
  \matrix (m) [matrix of math nodes,
               row sep=2em, column sep=2em, minimum width=2em]{
    A & \\
    A + B & X \\
    B & \\};
  \path[-stealth]
    (m-1-1) edge node [above] {$p$} (m-2-2)
    (m-3-1) edge node [below] {$q$} (m-2-2)
    (m-2-1) edge node [above] {$m$}   (m-2-2)
    (m-1-1) edge node [left] {$left$} (m-2-1)
    (m-3-1) edge node [left] {$right$} (m-2-1);
\end{tikzpicture}
\end{center}

Next we need prove the uniqueness of $m$. Suppose there exists another arrow $A + B \arrowto{h} X$, also makes the diagram commutes:

\[
h \circ left = p \quad \text{and} \quad h \circ right = q
\]

Taking any $a \in A, b \in B$, we have:

\[
\begin{cases}
h\ (a, 0) = h(left(a)) = (h \circ left)(a) = p(a) = m\ (a, 0) \\
h\ (b, 1) = h(right(a)) = (h \circ right)(b) = q(b) = m\ (b, 1)
\end{cases}
\]

Hence $h = m$, which proves the uniqueness of $m$.
\end{proof}

\ifx\wholebook\relax \else

\expandafter\enddocument
%\end{document}

\fi
